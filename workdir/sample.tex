\documentclass[uplatex,11pt]{jsarticle}

\usepackage{amsmath}
\usepackage{amssymb}
\usepackage[dvipdfmx]{graphicx}
\usepackage[dvipdfmx]{color}
\usepackage{here}
\usepackage{lipsum} % for dummy text

\title{サンプル文章}
\author{著者名}

\begin{document}

\maketitle

\section{はじめに}
この文書は日本語で書かれたLaTeXサンプルです。

\lipsum[1] % Generate some dummy text

上記のテキストは自動生成されたダミーです。
ダミーの確認をして、正常にコンパイルされていることを確認して下さい。
確認できましたか?

よかったです。。。

\section{メインセクション}
ここにメインの内容を書きます。適切なセクションに分けて内容を記述してください。


数式の例を以下に示します。

\begin{equation}
E = mc^2
\end{equation}

これはアインシュタインの質量エネルギー等価性の法則です。

\section{結論}
このように、LaTeXを用いて高品質な文書を作成することができます。
\section*{Dummy Section for Citation}
This is a dummy citation \cite{dummy2023}.

\bibliographystyle{plain}
\bibliography{sample_bib}
\end{document}
